% Note that the a4paper option is mainly intended so that authors in
% countries using A4 can easily print to A4 and see how their papers will
% look in print - the typesetting of the document will not typically be
% affected with changes in paper size (but the bottom and side margins will).
% Use the testflow package mentioned above to verify correct handling of
% both paper sizes by the user's LaTeX system.
%
% Also note that the "draftcls" or "draftclsnofoot", not "draft", option
% should be used if it is desired that the figures are to be displayed in
% draft mode.
%
\documentclass[conference]{IEEEtran}

\begin{document}
% paper title
% can use linebreaks \\ within to get better formatting as desired
\title{To \texttt{INT\_MAX}\ldots and beyond!
Exploring large-count support in MPI}

% author names and affiliations
% use a multiple column layout for up to three different
% affiliations
\author{\IEEEauthorblockN{Michael Shell}
\IEEEauthorblockA{School of Electrical and\\Computer Engineering\\
Georgia Institute of Technology\\
Atlanta, Georgia 30332--0250\\
Email: http://www.michaelshell.org/contact.html}
\and
\IEEEauthorblockN{Homer Simpson}
\IEEEauthorblockA{Twentieth Century Fox\\
Springfield, USA\\
Email: homer@thesimpsons.com}
\and
\IEEEauthorblockN{James Kirk\\ and Montgomery Scott}
\IEEEauthorblockA{Starfleet Academy\\
San Francisco, California 96678-2391\\
Telephone: (800) 555--1212\\
Fax: (888) 555--1212}}

% conference papers do not typically use \thanks and this command
% is locked out in conference mode. If really needed, such as for
% the acknowledgment of grants, issue a \IEEEoverridecommandlockouts
% after \documentclass

% use for special paper notices
%\IEEEspecialpapernotice{(Invited Paper)}

\maketitle

\begin{abstract}
%\boldmath
The abstract goes here.
\end{abstract}

% !TEX root = ../bigmpi.tex

\section{Introduction}

The Message Passing Interface~\cite{mpiforum:94, mpiforum:96, mpiforum:09, mpiforum:12} 
defines a broad set of functionality for writing parallel programs, especially across
distributed computing systems.

Now more than 20 years old, MPI continues to be widely used and has met the challenges of
post-petascale computing, including scaling to millions of cores CITE STUFF.

One area where MPI is less extensive than desirable is operations on very large data.
Specifically, MPI uses the native integer in the C (\texttt{int}) and Fortran (\texttt{INTEGER})
interfaces to represent the number of data elements to be communicated;
in some cases, this type is also used to express an offset (more on this later).  
This paper focuses on the issues with the C interface and we use the
well-known convention I$n_{I}$L$n_{L}$P$n_{P}$ to refer to the sizes
of the C types \texttt{int}, \texttt{long}, and \texttt{void*}, respectively.
For ILP32 systems, the largest buffer one can allocate is $2^{32}$ bytes (4GiB),
while MPI can handle buffers of up to 2GiB; the factor of two difference is
almost never a problem since 4GiB of \texttt{int} for example requires only
a count of only $2^30$.
A problem emerges in IL32P64 and I32LP64 systems because it is possible to allocate
more memory in a buffer than can be captured with an integer count and built-in datatype.
For example, a vector of 3 billion floats requires 12 GB of memory but cannot be 
communicated using any communication routine using built-in datatypes.



\bibliographystyle{IEEEtran}
\bibliography{IEEEabrv,bib/mpi}

\end{document}


