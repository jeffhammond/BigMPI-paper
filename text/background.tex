% !TEX root = ../bigmpi.tex

\section{Background}

Interface to MPI for large messages, i.e. those where the count argument
exceeds \texttt{INT\_MAX} but is still less than \texttt{SIZE\_MAX}.
BigMPI is designed for the common case where one has a 64b address
space and is unable to do MPI communication on more than $2^{31}$ elements
despite having sufficient memory to allocate such buffers.
BigMPI does not attempt to support large-counts on systems where
C \texttt{int} and \texttt{void\*} are both 32b.

Motivation

The MPI standard provides a wide range of communication functions that
take a C \texttt{int} argument for the element count, thereby limiting this
value to \texttt{INT\_MAX} or less.
This means that one cannot send, e.g. 3 billion bytes using the \texttt{MPI\_BYTE} 
datatype, or a vector of 5 billion integers using the \texttt{MPI\_INT} type, as
two examples.
There is a natural workaround using MPI derived datatypes, but this is
a burden on users who today may not be using derived datatypes.

This project aspires to make it as easy as possible to support arbitrarily
large counts ($2^{63}$ elements exceeds the local storage capacity of computers
for the foreseeable future).

This is an example of the code change required to support large counts using
BigMPI:
\begin{verbatim}
#ifdef BIGMPI
    MPIX_Bcast_x(stuff, large_count /* MPI_Count */, MPI_BYTE, 0, MPI_COMM_WORLD);
#else // cannot use count>INT_MAX
    MPI_Bcast(stuff, not_large_count /* int */, MPI_BYTE, 0, MPI_COMM_WORLD);
#endif
\end{verbatim}
