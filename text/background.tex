% !TEX root = ../bigmpi.tex

\section{Background}

THIS PROBABLY DOES NOT BELONG HERE

Functions that were easy:

- all variants of send and receive

- bcast, gather, scatter, allgather, alltoall (blocking and nonblocking)

- RMA (put, get, accumulate, get\_accumulate)

Each of these functions follows the pattern shown in 

\begin{figure}
\begin{code}
int MPIX_Send_x(const void *buf, MPI_Count count,
	MPI_Datatype datatype, int dest, int tag,
	MPI_Comm comm)
{
    int rc = MPI_SUCCESS;

    if (likely (count <= bigmpi_int_max )) {
        rc = MPI_Send(buf, (int)count, datatype,
	dest, tag, comm);
    } else {
        MPI_Datatype newtype;
        MPIX_Type_contiguous_x(count, datatype, &newtype);
        MPI_Type_commit(&newtype);
        rc = MPI_Send(buf, 1, newtype, dest, tag, comm);
        MPI_Type_free(&newtype);
    }
    return rc;
}
\end{code}
\label{code:mpi_send_x}
\end{figure}

% Please see src/type_contiguous_x.c in BigMPI.

In the general case, MPI\_Type\_create\_struct is required, although BigMPI tries to factorize the count into C integers so we can use MPI\_Type\_vector.

Ticket 423 would improve user experience because it addresses the common case of large counts of built-ins.

\subsection{Reductions}

Reductions were not easy

User-defined ops required for user-defined types.  This would be fixed by tickets 34+338.

How do I support MPI\_IN\_PLACE inside of my user-defined reduction?
I want to use MPI\_Reduce\_local?

Blocking reductions can use the cleaver.

\subsection{Vector-argument collectives}

All V-collectives turn into ALLTOALLW because different counts implies different large-count types.

ALLTOALLW displacements given in bytes are C int and therefore it is impossible to offset more than 2GB into the buffer.

NEIGHBOR\_ALLTOALLW to the rescue?!?!

MPI\_Neighbor\_alltoallw displacements are MPI\_Aint not int.  This is good.

Neighborhood collectives require special communicators that must be created for each call (and possibly cached).

Must allocate new argument vectors and, in the case of ?!alltoall?, we wastefully splat the same value in all locations.

% V-collectives using P2P and RMA

One can follow the definition in MPI to implement all of the V-collectives using P2P.

RMA (with win\_fence synchronization) also works for the V-collectives.

Allgatherv using nproc calls to Bcast also works.

Large-count definitely outside of recursive doubling regime so little to optimize...

% V-collectives - nonblocking issues

None of the aforementioned solutions works for nonblocking because:

What request do we return in the case of P2P or RMA?

Cannot free argument vectors until complete.

Any solution involving generalized requests is untenable for users.  BigMPI might use it.

\subsection{Neighborhood collectives}

Scalar collectives are easy.

V-collectives: map to ALLTOALLW

Same problem as before with nonblocking regarding the allocated argument vectors.

If not for MPI\_Aint displacements in ALLTOALLW, we would have to drop into P2P and MPICH generalized requests.
