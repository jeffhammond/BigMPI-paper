% !TEX root = ../bigmpi.tex

\section{Design}

The API follows the pattern of \texttt{MPI\_Type\_size(\_x)} in that all BigMPI
functions are identical to their corresponding MPI ones except that
they end with \texttt{\_x} to indicate that the count arguments have the type
\texttt{MPI\_Count} instead of \texttt{int}.
BigMPI functions use the MPIX namespace because they are not in the
MPI standard.


\texttt{MPIX\_Type\_contiguous\_x}
does the heavy lifting.  It's pretty obvious how it works.
A quality datatype engine will turn this into a contiguous datatype internally 
and thus the underlying communication will be efficient.  
This approach assumes a count-safe MPI implementation, but implementations need
to be count-safe period if the Forum is serious about datatypes being
the solution rather than \texttt{MPI\_Count} everywhere.

All of the communication functions follow the same pattern, demonstrated in
Figure~\ref{code:mpi_send_x}.
\end{figure}

BigMPI is optimized for the common case when count is smaller than $2^{31}$
with a \texttt{likely\_if} macro to minimize the performance hit for
this more common use case.  Our aim is for users to call the BigMPI routines
directly, instead of inserting a branch for the large-count case themselves.

To optimize further, one could implement
\texttt{MPIX\_Type\_contiguous\_x} using internals of the MPI implementation
instead of calling six MPI datatype functions.

\subsection{Limitations}

Even though \texttt{MPI\_Count} might be 128b, BigMPI only supports
64b counts (because of \texttt{MPI\_Aint} limitations and a desire to use \texttt{size\_t}
in unit tests), so BigMPI is not going to solve your problem if you
want to communicate more than 8 EiB of data in a single message.
Such computers do not exist nor is it likely that they will exist
in the foreseeable future.

BigMPI only supports built-in datatypes.  Code already using
derived-datatypes should already be able to handle large
counts without BigMPI \todo{talk about HINDEXED\_X in ROMIO}

Support for \texttt{MPI\_IN\_PLACE} is not implemented in some cases and
implemented inefficiently in others.
Using \texttt{MPI\_IN\_PLACE} is discouraged at the present time.
We hope to support it more effectively in the future.

BigMPI requires C99.  The BigMPI developers do not forsee systems with large
amount of memory or disk space also offering a compiler without this level of language support.

BigMPI only has C bindings right now.
Fortran 2003 bindings are planned.
Users desiring C++ bindings are encouraged to contact the BigMPI project.
