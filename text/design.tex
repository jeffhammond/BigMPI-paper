% !TEX root = ../bigmpi.tex

\section{Design}


MPI point-to-point routines, both two-sided and one-sided, were
trivial to implement via MPI datatypes.  This class of routines provides the
most commonly used MPI functionality, so for many codes the Forum has been
proven correct.  However, as we will see in Section~\ref{sec:reductions},
not all parts of the MPI standard will be so straightforward.

BigMPI implements all variants of send and receive, blocking and nonblocking variants of
the homogeneous collectives (bcast, gather, scatter, allgather, alltoall)
and RMA (put, get, accumulate, get\_accumulate)
along the lines of the example for MPI\_Send, shown in Figure~\ref{code:mpi_send_x}.

\begin{figure}
\begin{code}
int MPIX_Send_x(const void *buf, MPI_Count count,
                MPI_Datatype dt, int dest,
                int tag, MPI_Comm comm)
{
    int rc = MPI_SUCCESS;
    if (likely (count <= INT_MAX )) {
        rc = MPI_Send(buf, (int)count, dt, dest, tag, comm);
    } else {
        MPI_Datatype newtype;
        MPIX_Type_contiguous_x(count, dt, &newtype);
        MPI_Type_commit(&newtype);
        rc = MPI_Send(buf, 1, newtype, dest, tag, comm);
        MPI_Type_free(&newtype);
    }
    return rc;
}
\end{code}
\caption{The implementation of large-count Send, which serves as a template
for many other MPI-3 routines.\label{code:mpi_send_x}}
\end{figure}

The critical function in all of the large-count implementations noted above
is \texttt{MPIX\_Type\_contiguous\_x}, which emits a single datatype that
represents up to \texttt{SIZE\_MAX} elements.
%Supporting more elements than can fit into $2^{63}$ bytes of memory is
%not necessary, since such a system is unlikely to exist in the foreseeable future.
%However, since we now express these parameters in our own datatypes, we introduce
%a degree of flexibility not currently present in the MPI standard.
This utility routine allows us to implement large-count support in a straightforward
fashion since all instances of $(large\_count,type)$ are mapped to $(1,large\_type)$
by this function.
An associated decoder function extracts the original $large\_count$ from a
user-defined datatype; this function is employed within the user-defined reduction
operations.  Decoding a datatype is nontrivial even for such a simple case --
we must call \texttt{MPI\_Type\_get\_envelope} and \texttt{MPI\_Type\_get\_contents}
three times each just to unwind the result of \texttt{MPIX\_Type\_contiguous\_x}.
BigMPI is boon to the majority of application programmers that are unfamiliar 
with such features in the MPI standard by virtue of hiding these details.

Other datatypes can be supported easily within BigMPI, but this is not a high
priority because the primary goal is to solve the large-count problem for users
that are not currently making use of derived datatypes.
A user that employs derived datatypes in their code already is likely capable
of implementing their own large-count support already.
Nonetheless, the release version of BigMPI will support large-count equivalents
of all of the existing datatype constructors.

\begin{figure}
\begin{code}
int MPIX_Type_contiguous_x(MPI_Count count, 
                           MPI_Datatype oldtype, 
                           MPI_Datatype * newtype)
{
    assert(count<SIZE_MAX); /* has to fit into MPI_Aint */
    MPI_Count c = count/INT_MAX, r = count%INT_MAX;

    MPI_Datatype chunks, remainder;
    MPI_Type_vector(c, INT_MAX, INT_MAX, oldtype, &chunks);
    MPI_Type_contiguous(r, oldtype, &remainder);

    MPI_Aint lb /* unused */, extent;
    MPI_Type_get_extent(oldtype, &lb, &extent);

    MPI_Aint remdisp          = (MPI_Aint)c*INT_MAX*extent;
    int blklens[2]            = {1,1};
    MPI_Aint disps[2]         = {0,remdisp};
    MPI_Datatype types[2]     = {chunks,remainder};
    MPI_Type_create_struct(2, blklens, disps, types, newtype);

    MPI_Type_free(&chunks);
    MPI_Type_free(&remainder);

    return MPI_SUCCESS;
}
\end{code}
\label{code:type_contig_x}
\caption{Function for construction a large-count contiguous datatype.
A vector type describes a series of adjacent chunks, and a struct type picks up
any remaining data in case the count is not evenly divisible.}
\end{figure}

%In the general case, MPI\_Type\_create\_struct is required, although BigMPI tries to factorize the count into C integers so we can use MPI\_Type\_vector.

%Ticket 423 would improve user experience because it addresses the common case of large counts of built-ins.

% !TEX root = ../bigmpi.tex

\subsection{Reductions}
\label{sec:reductions}

Large-count support for reductions poses a challenge, particularly in the nonblocking case.
For the blocking case, it is straightforward to block a single large-count operation into
multiple normal-count (count$<2^{31}$) operations (we will refer to this as chunking); 
however, as it is not possible to return a single request object associated with more than 
one nonblocking operation, we cannot implement nonblocking reductions in this manner.
Generalized requests - the MPI-standard way to implement non-blocking operations 
in a library - are not a viable alternative for reasons that have been documented 
in other work (\cite{latham:grequest-extensions}).
For the blocking case, it is desirable to use chunking because many MPI implementations 
have optimized implementations of reductions for built-in reduction operations.

The MPI standard stipulates that built-in reduction operations can be used with built-in types
in the case of reductions, which means that performing a reduction on a vector of $N$
doubles using count=$N$ and type=\texttt{MPI\_DOUBLE} is compatible with \texttt{MPI\_SUM},
whereas the same reduction performed using a contiguous datatype to represent the vector
of doubles requires a user-defined reduction operation.
Thus, BigMPI creates user-defined operations corresponding to all the built-in reductions
acting on contiguous datatypes.  Inside of these reduction operations, the datatype is
decoded and the reduction performed using multiple calls to \texttt{MPI\_Reduce\_local}
and the appropriate built-in reduction operation.
This is a general solution that works for both the blocking and nonblocking cases,
at least for out-of-place reductions.

Unfortunately, user-defined reductions cannot support \texttt{MPI\_IN\_PLACE}.
The user-defined reduce function interface (see below) does not expose
the information required to do an arbitrary in-place reduction.
\begin{code}
MPI_User_function(void* invec, void* inoutvec, 
                  int *len, MPI_Datatype *datatype);
\end{code}
As user-defined reduce operations are the only way to implement
large-count nonblocking reductions, \textit{we identify this as the first example
where MPI-3 lacks the necessary features to support large counts effectively,
as the inefficiency associated with user-defined reductions and lack of support
for in-place reductions has a substantial negative impact on users.}
% !TEX root = ../bigmpi.tex

\subsection{Vector-argument collectives}

Vector-argument collectives (henceforth v-collectives) are the generalization of, 
for example, \texttt{MPI\_Scatter}, \texttt{MPI\_Gather}, and \texttt{MPI\_Alltoall}
when the count but not the datatype varies across processes.
When datatypes are used to support large counts, all  these operations must be
mapped to \texttt{MPI\_Alltoallw} because each large count will be mapped
to a different user-defined datatype, and \texttt{MPI\_Alltoallw} is the only collective
that supports a vector of datatypes.
Using \texttt{MPI\_Alltoallw} to implement, for example, a large-count \texttt{MPI\_Scatterv} is
particularly inefficient because the former assumes inputs from every process,
whereas the latter uses only the input from the root.
However, the overhead of scanning a vector of counts where all but one is zero
is almost certainly inconsequential compared with the cost of transmitting a buffer 
of $2^{31}$ bytes.
%Additionally, such vectors are unlikely to be particularly long since scattering
%a buffer that would run into large-count issues r

%All V-collectives turn into ALLTOALLW because different counts implies different large-count types.

The v-collectives encounter a second, more subtle issue due to the mapping to
\texttt{MPI\_Alltoallw}.  Because this function takes a vector of datatypes, the
displacements into the input and output vectors are given in bytes,
not element count, and the type of this offset is a C integer.
This creates an overflow situation \textit{even 
when the input buffer is less than 2 GiB} because a vector of 1 billion
alternating integers and floats may require an byte offset in excess of $2^{31}$.
Thus, \texttt{MPI\_Alltoallw} is not an acceptable solution for the large-count
v-collectives because of the likelihood of overflowing in the displacement vector.
The use of the  C integer instead of \texttt{MPI\_Aint} for the displacement vector in
the  collective operations added prior to MPI-3 is an unfortunate oversight that
cannot be rectified without breaking backward compatibility.

%ALLTOALLW displacements given in bytes are C int, and therefore it is impossible to offset more than 2GB into the buffer.
%NEIGHBOR\_ALLTOALLW to the rescue?!?!
%MPI\_Neighbor\_alltoallw displacements are MPI\_Aint not int.  This is good.
%Neighborhood collectives require special communicators that must be created for each call (and possibly cached).
%Must allocate new argument vectors and, in the case of alltoall, we wastefully splat the same value in all locations.

Fortunately, the overflow issue with displacements in \texttt{MPI\_Alltoallw} is
resolved by using the neighborhood collectives introduced in MPI-3, which do
use \texttt{MPI\_Aint} for displacements.
On the other hand, neighborhood collectives require an appropriate
communicator, which must be constructed prior to calling \texttt{MPI\_Neighborhood\_alltoallw}.
BigMPI creates a distributed graph communicator using \texttt{MPI\_Dist\_graph\_adjacent}
on the fly for every invocation of the large-count v-collectives, which instead are assumed to incur
insignificant overhead compared with the data movement entailed in such an operation.
It is straightforward to optimize for the common cases of \texttt{MPI\_COMM\_WORLD} for
non-rooted collectives and \texttt{MPI\_COMM\_WORLD} with \texttt{root=0}, but this 
is not currently implemented.

The implementation of large-count v-collectives using \texttt{MPI\_Neighborhood\_alltoallw} 
requires two $O(n_{proc})$ setup steps. The first allocates and populates the vectors of 
send and receive counts, displacements, and datatypes. 
The second creates a distributed graph communicator.
Figures~\ref{code:BigMPI_Convert_vectors} and~\ref{code:BigMPI_Create_graph_comm}
show the implementation of these functions, which are included in their entirety to illustrate that
although the mapping from v-collectives to \texttt{MPI\_Neighborhood\_alltoallw} is feasible,
it is rather involved and in some cases unnatural.
Creating the vector of datatypes requires $O(n_{proc})$ calls to \texttt{BigMPI\_Type\_contiguous\_x},
which itself requires six MPI calls, although all of these are expected to be inexpensive.

% V-collectives using P2P and RMA
%One can follow the definition in MPI to implement all of the V-collectives using P2P.
%RMA (with win\_fence synchronization) also works for the V-collectives.
%Allgatherv using nproc calls to Bcast also works.
%Large-count definitely outside of recursive doubling regime so little to optimize...

An alternative approach to implementing large-count v-collectives is to map
these to point-to-point operations, although this works only for blocking operations
because of the inability to aggregate requests, as described above.
Since large-count v-collectives are well outside the regime where latency-oriented 
optimizations such as recursive-doubling are important, this approach is unlikely to have a significant impact 
on performance, and it eliminates the need for some of the $O(n_{proc})$ setup steps.
The MPI standard describes every collective in terms of its implementation 
in terms of send-recv calls; the point-to-point BigMPI implementation 
follows these recipes closely:
(1) nonblocking receives are preposted by the root or all ranks as appropriate;
(2) the root or all ranks then call nonblocking send; 
and (3) all ranks then call Waitall.
Since the large-count BigMPI send-recv functions are used, there is no need for
$O(n_{proc})$ vectors of datatypes, and so forth---only a vector of \texttt{MPI\_Request}
objects for the nonblocking operations is required.

A third implementation of v-collectives is to use RMA (one-sided) that follows
the same traffic pattern as the point-to-point implementation.
In this case, an MPI window must be created associated with the source (target)
buffers and \texttt{MPI\_Get} (\texttt{MPI\_Put}) operations used for moving data.
The most appropriate synchronization mode for mapping collectives to RMA
is \texttt{MPI\_Win\_fence}, although one could use a passive target instead.
If a future version of the MPI standard introduces a nonblocking
equivalent of \texttt{MPI\_Win\_fence} or \texttt{MPI\_Win\_unlock\_all}, these
could be used to implement nonblocking v-collectives in terms of RMA; 
at least within MPI-3, we are limited to the blocking case.
The RMA implementation was prototyped in BigMPI but is not currently implemented.
%because no performance benefit of one-sided is expected because of the current state
The current state of RMA implementations map one-sided operations to two-sided ones internally. Thus we would expect to see no performance benefit from BigMPI's RMA approach.
If RMA operations exploit RDMA hardware, 
however, noticeable performance improvements may be
observed.

While not named as such, \texttt{MPI\_Reduce\_scatter} is a v-collective.
BigMPI currently does not yet support this function,
but it is straightforward to implement in terms of  \texttt{MPI\_Reduce} and
\texttt{MPI\_Scatterv}, which will be the basis for the BigMPI implementation.

% V-collectives - nonblocking issues

%None of the aforementioned solutions works for nonblocking because:
%What request do we return in the case of P2P or RMA?
%Cannot free argument vectors until complete.
%Any solution involving generalized requests is untenable for users.  BigMPI might use it.

Unfortunately, nonblocking v-collectives cannot be implemented by using the aforementioned approaches.
In the case of the neighborhood collective
implementation, we cannot free the vector temporaries holding the counts,
displacements, and datatypes until the operation has completed.
If callback functions associated with request completion were present in the
MPI standard (see \cite{ticket26} for a proposal of this), then it would
be possible to free the temporary buffers using this callback.
Since one cannot associate a single request with multiple
nonblocking operations, the point-to-point implementation is not viable
for the nonblocking v-collectives.
Moreover, all relevant forms of MPI RMA synchronization have blocking semantics
and thus cannot be used to implement nonblocking collectives.

\textit{We identify nonblocking v-collectives as the second example
where MPI-3 lacks the necessary features to support large counts.}

\begin{figure}
\begin{code}
void BigMPI_Convert_vectors(int num,
                            int splat_old_count,
                            const MPI_Count oldcount,
                            const MPI_Count oldcounts[],
                            int splat_old_type,
                            const MPI_Datatype oldtype,
                            const MPI_Datatype oldtypes[],
                            int zero_new_displs,
                            const MPI_Aint olddispls[],
                            int newcounts[],
                            MPI_Datatype newtypes[],
                            MPI_Aint newdispls[])
{
    assert(splat_old_count || (oldcounts!=NULL));
    assert(splat_old_type  || (oldtypes!=NULL));
    assert(zero_new_displs || (olddispls!=NULL));

    MPI_Aint lb /* unused */, oldextent;
    if (splat_old_type) {
        MPI_Type_get_extent(oldtype, &lb, &oldextent);
    } else {
        /* !splat_old_type implies ALLTOALLW, 
            which implies no displacement zeroing. */
        assert(!zero_new_displs);
    }

    for (int i=0; i<num; i++) {
        /* counts */
        newcounts[i] = 1;

        /* types */
        MPIX_Type_contiguous_x(oldcounts[i], 
                       splat_old_type ? oldtype : oldtypes[i], 
                       &newtypes[i]);
        MPI_Type_commit(&newtypes[i]);

        /* displacements */
        MPI_Aint newextent;
        /* If we are not splatting old type, it implies 
         *  ALLTOALLW, which does not scale the 
         * displacement by the type extent,
         * nor would we ever zero the displacements. */
        if (splat_old_type) {
            MPI_Type_get_extent(newtypes[i], &lb, &newextent);
            newdispls[i] = (zero_new_displs ? 0 : 
                            olddispls[i]*oldextent/newextent);
        } else {
            newdispls[i] = olddispls[i];
        }
    }
    return;
}
\end{code}
\caption{Function for populating the vector inputs 
for \texttt{MPI\_Neighborhood\_alltoallw} for the various v-collectives.}
\label{code:BigMPI_Convert_vectors}
\end{figure}

\begin{figure}
\begin{code}
int BigMPI_Create_graph_comm(MPI_Comm comm_old, int root, 
                             MPI_Comm * comm_dist_graph)
{
    int rank, size;
    MPI_Comm_rank(comm_old, &rank);
    MPI_Comm_size(comm_old, &size);

    /* in the all case (root == -1), every rank is a 
     * destination for every other rank;
     * otherwise, only the root is a destination. */
    int indeg  = (root == -1 || root==rank) ? size : 0;
    /* in the all case (root == -1), every rank is a 
     * source for every other rank;
     * otherwise, all non-root processes are the 
     * source for only one rank (the root). */
    int outdeg = (root == -1 || root==rank) ? size : 1;

    int * srcs = malloc(indegree*sizeof(int));  
    assert(srcs!=NULL);
    int * dsts = malloc(outdegree*sizeof(int)); 
    assert(dsts!=NULL);

    for (int i=0; i<indegree; i++) {
        srcs[i] = i;
    }
    for (int i=0; i<outdegree; i++) {
        dsts[i] = (root == -1 || root==rank) ? i : root;
    }

    int empty = MPI_WEIGHTS_EMPTY;
    int unwtd = MPI_UNWEIGHTED;
    int rc = MPI_Dist_graph_create_adjacent(comm_old,
                indeg, srcs, indeg==0 ? empty : unwtd,
                outdeg, dsts, outdeg==0 ? empty : unwtd,
                MPI_INFO_NULL, 0 /* reorder */, 
                comm_dist_graph);

    free(srcs);
    free(dsts);

    return rc;
}
\end{code}
\caption{Function for constructing the distributed graph communicator
that allows the mapping of both rooted (e.g. \texttt{MPI\_Gatherv}) and
non-rooted (e.g. \texttt{MPI\_Allgatherv}) collectives to
\texttt{MPI\_Neighborhood\_alltoallw}.}
\label{code:BigMPI_Create_graph_comm}
\end{figure}

% !TEX root = ../bigmpi.tex

\subsection{Neighborhood collectives}

%Scalar collectives are easy.
%V-collectives: map to ALLTOALLW
%Same problem as before with nonblocking regarding the allocated argument vectors.
%If not for MPI\_Aint displacements in ALLTOALLW, we would have to drop into P2P and MPICH generalized requests.

The implementation of large-count neighborhood collectives is straightforward
using the approach noted above for mapping v-collectives to \texttt{MPI\_Neighborhood\_alltoallw},
except that we omit the creation of the distributed graph communicator.
All the issues with the nonblocking cases still exist, since temporary vectors
are still required for the mapping of $(large\_count,type)$ to $(1,large\_type)$ for all ranks.
Thus, \textit{we identify nonblocking neighborhood collectives as the third example
where MPI-3 lacks the necessary features to support large counts.}


%%%%%%%%%%%%%%%%%%%%%%%%%%%%%%%%%%%%%%%%%%

\begin{comment}
The API follows the pattern of \texttt{MPI\_Type\_size(\_x)}: all BigMPI
functions are identical to their corresponding MPI ones except that
they end with \texttt{\_x} to indicate that the count arguments have the type
\texttt{MPI\_Count} instead of \texttt{int}.
BigMPI functions use the MPIX namespace because they are not in the
MPI standard.


\texttt{MPIX\_Type\_contiguous\_x}
does the heavy lifting.  It's pretty obvious how it works.
A quality datatype engine will turn this into a contiguous datatype internally 
and thus the underlying communication will be efficient.  
This approach assumes a count-safe MPI implementation, but implementations need
to be count-safe period if the Forum is serious about datatypes being
the solution rather than \texttt{MPI\_Count} everywhere.

All of the communication functions follow the same pattern, demonstrated in
Figure~\ref{code:mpi_send_x}.
\end{comment}

\subsection{Interface}

Our interface differs from standard MPI only in promoting the ``count'' 
of the (count, datatype) pair to the larger MPI\_Count type.

\subsection{Details}

BigMPI is optimized for the common case when count is smaller than $2^{31}$
with a \texttt{likely\_if} macro to minimize the performance hit for
this more common use case.  Our aim is for users to call the BigMPI routines
directly, instead of inserting a branch for the large-count case themselves.

To optimize further, one could implement
\texttt{MPIX\_Type\_contiguous\_x} using internals of the MPI implementation
instead of calling six MPI datatype functions.  Considering we are operating on
gigabytes of data, the performance gains one might wrest from such an approach
would be limited.

The software implementation is found in the BSD-licensed BigMPI
library and is relatively friendly to users, i.e.\ it has both a Cmake
and an Autotools build system and requires a generic programming
environment composed of a C99 compiler and an implementation
of MPI-3.

\subsection{Limitations}

Even though \texttt{MPI\_Count} might be 128b, BigMPI only supports
64b counts (because of \texttt{MPI\_Aint} limitations and a desire to use \texttt{size\_t}
in unit tests), so BigMPI is not going to solve your problem if you
want to communicate more than 8 EiB of data in a single message.
Such computers do not exist nor is it likely that they will exist
in the foreseeable future.

BigMPI only supports built-in datatypes.  Code already using
derived-datatypes should already be able to handle large counts without BigMPI\@.
However, see Section~\ref{sec:romio_typeproc} for an example of HINDEXED not
being sufficient.

Support for \texttt{MPI\_IN\_PLACE} is not implemented in some cases and
implemented inefficiently in others.
Using \texttt{MPI\_IN\_PLACE} is discouraged at the present time.
We hope to support it more effectively in the future.

BigMPI requires C99.  The BigMPI developers do not foresee systems with large
amount of memory or disk space also offering a compiler without this level of language support.

BigMPI only has C bindings right now.
Fortran 2003 bindings are planned.
Users desiring C++ bindings are encouraged to contact the BigMPI project.
