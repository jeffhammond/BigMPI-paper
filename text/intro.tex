% !TEX root = ../bigmpi.tex

\section{Introduction}

The Message Passing Interface~\cite{mpiforum:94, mpiforum:96, mpiforum:09, mpiforum:12} 
defines a broad set of functionality for writing parallel programs, especially across
distributed computing systems.

Now more than 20 years old, MPI continues to be widely used and has met the challenges of
post-petascale computing, including scaling to millions of cores CITE STUFF.



MPI-3 largely punted on large-count support

A handful of MPI\_Foo\_x routines make rudimentary large-count support possible

Forum asserted that ``just use datatypes'' is a sufficient solution for users.

Obviously, no one bothered to verify the aforementioned assertion...

BigMPI provides a high-level library that supports large counts that can be used as 
drop-in replacement for existing MPI routines and test the Forum?s assertion 
that datatypes are sufficient for large-count support.



Interface to MPI for large messages, i.e. those where the count argument
exceeds \texttt{INT\_MAX} but is still less than \texttt{SIZE\_MAX}.
BigMPI is designed for the common case where one has a 64b address
space and is unable to do MPI communication on more than $2^{31}$ elements
despite having sufficient memory to allocate such buffers.
BigMPI does not attempt to support large-counts on systems where
C \texttt{int} and \texttt{void\*} are both 32b.

\subsection{Motivation}

The MPI standard provides a wide range of communication functions that
take a C \texttt{int} argument for the element count, thereby limiting this
value to \texttt{INT\_MAX} or less.
This means that one cannot send, e.g. 3 billion bytes using the \texttt{MPI\_BYTE} 
datatype, or a vector of 5 billion integers using the \texttt{MPI\_INT} type, as
two examples.

These limitations may seem academic in nature, as 2 billion
\texttt{MPI\_DOUBLE} equate to 16GB and one might think that
applications may rarely ever need to transmit that much data, as there
may be less RAM available for the whole NUMA domain in which the MPI
process is running. Two recent trends may render this limit
increasingly impractical: first, growing core counts per CPU mean
larger data portions per MPI process and second, Big Data applications
may demand more RAM per core than the traditional 2GB for computer
simulations.

If the user code manually packs data, either for
performance~\cite{jenkins2012enabling} or encoding
reasons~\cite{boostmpi}, then the MPI implementation may be given just
an array of \texttt{MPI\_BYTE}, which further reduces the maximum
message size (e.g. 250 million for \texttt{MPI\_DOUBLE}).

A natural workaround is to use MPI derived datatypes. While it is
plausible that application developers will know typical data sizes and
can thus intercept calls which may exceed the \texttt{INT\_MAX} limit,
another scenario is harder to solve: problem solving
environments~\cite{cactus:SC01, gromacs} and computational
libraries~\cite{physis, libgeodecomp} operate on data structures with
user-defined dimensions. To ensure correctness, developers would need
to safeguard all communication functions which operate on user data.

As we will demonstrate in the following sections, this is not a
trivial task for all communication operations. A generic, reusable
solution as provided by BigMPI alleviates the required implementation
effort; library users can use preprocessor directives in a header file
to select the appropriate set of communication functions:

\begin{code}
// configuration header:
#ifdef BIGMPI
#define MPIX_Bcast_x MPIX_Bcast_x
#define MPIX_Send_x MPIX_Send_x
...
#else // cannot use count>INT_MAX
#define MPIX_Bcast_x MPI_Bcast
#define MPIX_Send_x MPI_Send
...
#endif

// usage:
    MPIX_Bcast_x(buf, large_count, MPI_BYTE, 0, mycomm);
#endif
\end{code}

This project aspires to make it as easy as possible to support arbitrarily
large counts ($2^{63}$ elements exceeds the local storage capacity of computers
for the foreseeable future).

% This text sucks

% One area where MPI is less extensive than desirable is operations on very large data.
% Specifically, MPI uses the native integer in the C (\texttt{int}) and Fortran (\texttt{INTEGER})
% interfaces to represent the number of data elements to be communicated;
% in some cases, this type is also used to express an offset (more on this later).  

This paper focuses on the issues with the C interface and we use the
well-known convention I$n_{I}$L$n_{L}$P$n_{P}$ to refer to the sizes
of the C types \texttt{int}, \texttt{long}, and \texttt{void*}, respectively.
For ILP32 systems, the largest buffer one can allocate is $2^{32}$ bytes (4GiB),
while MPI can handle buffers of up to 2GiB; the factor of two difference is
almost never a problem since 4GiB of \texttt{int} for example requires only
a count of only $2^{30}$.

% A problem emerges in IL32P64 and I32LP64 systems because it is possible to allocate
% more memory in a buffer than can be captured with an integer count and built-in datatype.
% For example, a vector of 3 billion floats requires 12 GB of memory but cannot be 
% communicated using any communication routine using built-in datatypes.

