% !TEX root = ../bigmpi.tex

\section{Introduction}
\label{sec:intro}

The Message Passing Interface~\cite{mpiforum:94, mpiforum:96, mpiforum:09, mpiforum:12} 
defines a broad set of functionality for writing parallel programs, especially across
distributed computing systems.
Now more than 20 years old, MPI continues to be widely used and has met the challenges of
post-petascale computing, including scaling out to millions of cores~\cite{balaji2011mpi}. 
In order to scale up, in terms of problem size, one needs
to be able to describe large working sets.  The existing (count, datatype) pair
works well, until the ``count'' exceeds the range of the native integer type
(in the case of the C interface, \texttt{int}, which is 32 bits on most current platforms).
We call this the ``large-count'' problem.

When drafting MPI-3 the MPI Forum took a minimalist approach large-count 
support~\cite{ticket265}.
The forum introduced a handful of MPI\_Foo\_x routines that provide a large-count
equivalent of an existing MPI\_Foo to make rudimentary large-count support possible.
To be explicit, in this context, Foo is ``Get\_elements'', ``Type\_size'', ``Type\_get\_extent'', 
``Type\_get\_true\_extent'', ``Status\_set\_elements'', which is the minimal set
of functions that must support large counts in order for one to be able to deal 
with derived datatypes that represent large counts.
After lengthy deliberation, the Forum asserted that ``just use datatypes'' is 
a sufficient solution for users~\cite{squyres-blog-large-count}.
For example, one can describe 4 billion bytes as 1 billion 4-byte integers.
Or, one could use contiguous MPI dataypes to describe 16 billion bytes as 1,000
16 million-byte chunks.  For these simple examples, it is easy to envision a
solution. It is only through implementing the proposed approach for all cases
in MPI that one discovers the challenges hidden in such an assertion.

BigMPI provides a high-level library that attempts to support large counts.
It was written to test the Forum's assertion
that datatypes are sufficient for large-count support and to provide a drop-in library
solution for applications requiring large-count support.
In this context, ``large-count'' is any count that exceeds \texttt{INT\_MAX}.
BigMPI makes the smallest possible changes to the MPI standard routines to
enable large counts, minimizing application changes.

BigMPI is designed for the common case where one has a 64-bit address
space and is unable to do MPI communication on more than $2^{31}$ elements
despite having sufficient memory to allocate such buffers.
%BigMPI does not attempt to support large-counts on systems where
%C \texttt{int} and \texttt{void\*} are both 32 bits.
As systems with more than $2^{63}$ bytes (8192 PiB) of memory 
per node are unlikely to exist for the foreseeable future --
the total system memory capacity for an exascale machine has been 
predicted to be 50-100 petabytes~\cite{shalf2011exascale} --
BigMPI makes no attempt to support the full range of \texttt{MPI\_Count}
(possibly a 128-bit integer) internally; rather it uses \texttt{size\_t}
and \texttt{MPI\_Aint}, as these reflect the limit of the available memory
rather than the theoretical filesystem size (as \texttt{MPI\_Count} does).