% !TEX root = ../bigmpi.tex

\section{Related Work}

As noted in \S\ref{sec:intro} and \S\ref{sec:mpi4}, the MPI Forum has made efforts 
to address count-safety issues in the MPI standard.
Both MPICH and OpenMPI have made significant strides toward count-safety
at the implementation level.  MPICH currently passes all of the large-count tests
in its own test suite, although these tests may not exercise all possible code paths.
We are not aware of other efforts to implement a high-level library on top of
MPI-3 that supports large-count usage in the manner that BigMPI does.

\subsection{OpenSHMEM}

OpenSHMEM 1.0~\cite{kuehn2012openshmem} conscientiously uses \texttt{size\_t} for
counts and \texttt{ptrdiff\_t} for offsets throughout and hence is a count-safe API.
Since numerous implementations of OpenSHMEM exist, we cannot evaluate
the count-safety of all of them.
When a count-safe API such as DMAPP~\cite{DMAPP} is used, however, 
count-safety is more likely
than if the implementation is required to map from 64-bit counts to 32-bit counts internally.

\subsection{GASNet}

GASNet uses \texttt{size\_t} and is thus count-safe.  We have not attempted to evaluate
the count-safety of GASNet implementations, since there are numerous conduits, each of
which might have large-count issues due to platform-specific low-level APIs and
bugs in system software.

\subsection{GA/ARMCI}

Both the Global Arrays~\cite{nieplocha:94} and ARMCI~\cite{nieplocha:99} interfaces
use native integer types in both C and Fortran to represent element counts; and
in the case of ARMCI Put and Get, the count is in terms of bytes, not elements.
Thus, both models have the same (or worse) large-count issues as MPI-3.
