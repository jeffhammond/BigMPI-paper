% !TEX root = ../bigmpi.tex

\section{Making MPI Implementations Large-count Clean}
% MPICH dataloop code
As mentioned earlier, the MPI Forum contended that a ``count, type''
tuple was sufficient to describe arbitrarily large types.  As of mid 2013, few codes
required large count functionality, and those that did had devised workarounds.
BigMPI and the re-emergence of a certain class of I/O routines finally served
as the motivation needed to audit the MPICH code.  To illustrate the challenges in
making any MPI implementation ``large count'' clean, we describe the changes
needed for the MPICH datatype processing engine and the ROMIO I/O library. We
also share our experiences with an unfortunate OS limitation.


\subsection{MPICH Datatloop code}
The MPICH code base prior to the
3.1 release contained widespread assumptions that an int-sized type
would be sufficient to contain not only the size of a datatype but also the product
of a count of the number of datatypes and the size of those types.
Even before MPI-3, this assumption was false: the size of a million MPI\_DOUBLE
types exceeds 32 bits.
An obvious first step would be to promote `int' to `MPI\_Count' wherever it was
used to hold a size.
Out of concern for possibly conducting 128-bit math on a 64 bit platform (a
rather poorly performing situation on the LP64 machines common in 2014),
we instead used MPI\_Aint.  The MPI\_Aint type, large enough to hold a
count of bytes for a memory allocation, will be sufficient to describe the file
and memory use cases we envision.  To find all the locations in 8600 lines of
code requiring promotion, the Clang compiler warning flag
\texttt{-Wshorten-64-to-32} proved invaluable for this task.  The compiler option has
flagged many more
locations in the MPICH code that remain in need of examination.

% ROMIO size routines
\subsection{ROMIO type processing}
\label{sec:romio_typeproc}

Once we made it possible for MPICH to describe arbitrarily large datatypes, we
needed to update the ROMIO layer to understand these new larger datatypes.
ROMIO \cite{thakur:mpi-io-implement} was designed to be a portable
 implementation of MPI-2's I/O
chapter.  While in modern practice it is almost always part of an MPI
implementation, it is possible to build a stand-alone ROMIO library.  Thus,
ROMIO strives to use only MPI library routines to process datatypes, and not reach into the internal datatype processing engine of the underlying MPI implementation.

The MPI-3 standard provides the large-count aware \_x variants of
MPI\_Type\_get\_size, but
ROMIO, like the MPICH dataloop code, used int types for the count.
Here again,  we had to audit ROMIO for instances of storing $count * size$
into an integer, an operation that would result in the compiler truncating the
result upon assignment.

Even some surprising regions of ROMIO needed updating. For one example, the
two-phase collective buffering optimization will split up even large requests
into ``cb\_buffer\_size'' chunks.  However, there is still a preliminary step
where ROMIO exchanges offset-length pairs among coordinating processes.  ROMIO
constructs an HINDEXED type to describe these pairs. HINDEXED's ``lengths'' array
is defined as an int type.  ROMIO borrowed the BigMPI ideas and implemented an
HINDEXED datatype constructor that used an MPI\_Count type for its length array.


% UNIX system calls
\subsection{UNIX system calls}

Finally, after updating MPICH and ROMIO to accommodate large data transfers,
we are left with one last problem: the system call layer.
The write system call has the following prototype:

\begin{code}
ssize_t write(int fd, const void *buf, size_t count);
\end{code}

where size\_t is supposed to be big enough to hold ``the size of an object''
\cite{posix-std}.  However, we must remember that the rule for write is that it may
write ``up to count bytes''.  In practice, short writes to a file are not seen --
until the count of bytes approaches $2^{31}$.  On Linux, we observed 
the write system call outputting at most 
$2^{31}-4096$ bytes no matter how many bytes were requested,
necessitating the introduction of retry
logic.  On Darwin and BSD, the story is even worse: if $2^{31}$ bytes are passed
down to the read or write system call, the call will return an error.  We now
cap the size of a transfer to INT\_MAX and issue multiple system calls until
all bytes have been transfered.  The lesson for implementors is clear:
operating on large amounts of data has seen very little test coverage throughout the software stack.
