\subsection{Performance optimizations}

BigMPI is optimized for the case when count is smaller than $2^{31}$
with a \texttt{likely\_if} macro to minimize the performance hit for
the common case.  The aim is for users to call the BigMPI routines
directly, instead of inserting a branch for the large-count case themselves.
It is assumed that branch mis-prediction is significantly less expensive than
transferring gigabytes of data across the network.

While software overhead is expected to be insignificant compared to data movement
in BigMPI, it is possible to reduce the overhead of \texttt{MPIX\_Type\_contiguous\_x}
by implementing it using the internal functions of the MPI implementation,
which we have prototype within MPICH already 
(\href{https://github.com/jeffhammond/mpich/tree/type_contiguous_x}
{https://github.com/jeffhammond/mpich/tree/type\_contiguous\_x})
and begun prototyping within Open MPI.

Additional optimizations included caching graph communicators or windows associated
with v-collectives and searching count vectors for repetition to reduce the number of 
user-defined datatypes required.
The former optimization was previously implemented
in BigMPI but was removed because of the challenges associated with making it 
thread-safe and the goal to neither require a special initialization routine for BigMPI
nor intercept MPI's own initialization routine via PMPI interposition.

In general, the goal of BigMPI is to provide a straightforward implementation of
large-count support using a friendly library interface.
The best way to develop an optimized implementation of large-count support
is within an MPI implementation, whether that be through new functions in MPI-4
or non-standard extensions to MPI provided by a particular implementation.
For example, it would be straightforward, albeit a substantial amount of work,
to implement the large-count operations of the BigMPI interface within MPICH.



