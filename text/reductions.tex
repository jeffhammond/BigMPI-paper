% !TEX root = ../bigmpi.tex

\subsection{Reductions}
\label{sec:reductions}

Large-count support for reductions poses a challenge, particularly in the nonblocking case.
For the blocking case, it is straightforward to block a single large-count operation into
multiple normal-count (count$<2^{31}$) operations (we will refer to this as chunking); 
however, as it is not possible to return a single request object associated with more than 
one nonblocking operation, we cannot implement nonblocking reductions in this manner.
Generalized requests - the MPI-standard way to implement non-blocking operations 
in a library - are not a viable alternative for reasons that have been documented 
in other work (\cite{latham:grequest-extensions}).
For the blocking case, it is desirable to use chunking because many MPI implementations 
have optimized implementations of reductions for built-in reduction operations.

The MPI standard stipulates that built-in reduction operations can be used with built-in types
in the case of reductions, which means that performing a reduction on a vector of $N$
doubles using count=$N$ and type=\texttt{MPI\_DOUBLE} is compatible with \texttt{MPI\_SUM},
whereas the same reduction performed using a contiguous datatype to represent the vector
of doubles requires a user-defined reduction operation.
Thus, BigMPI creates user-defined operations corresponding to all the built-in reductions
acting on contiguous datatypes.  Inside of these reduction operations, the datatype is
decoded and the reduction performed using multiple calls to \texttt{MPI\_Reduce\_local}
and the appropriate built-in reduction operation.
This is a general solution that works for both the blocking and nonblocking cases,
at least for out-of-place reductions.

Unfortunately, user-defined reductions cannot support \texttt{MPI\_IN\_PLACE}.
The user-defined reduce function interface (see below) does not expose
the information required to do an arbitrary in-place reduction.
\begin{code}
MPI_User_function(void* invec, void* inoutvec, 
                  int *len, MPI_Datatype *datatype);
\end{code}
As user-defined reduce operations are the only way to implement
large-count nonblocking reductions, we identify this as the first example
where MPI-3 lacks the necessary features to support large counts.
Lack of support for \texttt{MPI\_IN\_PLACE} in user-defined reductions
is a related shortcoming of the MPI-3 standard.