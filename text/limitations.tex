\subsection{Limitations}

As previously stated, BigMPI does not actually support the full
range of \texttt{MPI\_Count}, but rather only the range of the
address space (i.e. \texttt{size\_t} and \texttt{MPI\_Aint}), as buffers
larger than the address space are rather difficult to allocate.

BigMPI only supports built-in datatypes.  
Code already using derived-datatypes should already be able 
to handle large counts without BigMPI\@.
However, see Section~\ref{sec:romio_typeproc} 
for an example of HINDEXED not being sufficient.

Support for \texttt{MPI\_IN\_PLACE} is not 
implemented in some cases (e.g. where it is impossible) and 
implemented inefficiently (i.e. via a buffer copy) in others.
Using \texttt{MPI\_IN\_PLACE} is discouraged at the present time
although it is expected that it will be supported efficiently whenever
possible in the release version of BigMPI.

BigMPI requires C99.  Fifteen years is more than enough time for compiler 
implementers interested in supporting ISO languages to provide a C99
compiler.

The MPI-3 standard supports language bindings for C and Fortran --
the latter via three different mechanisms
(\texttt{mpif.h}, \texttt{use mpi} and \texttt{use mpi\_f08}).
Currently, BigMPI only provides a C interface, but a Fortran 2003
interface to the C API via \texttt{ISO\_C\_BINDING} is planned.
It is expected that C++ programmers will be able to make use of the
C interface and can implement wrappers consistent with their own
style of C++ programming.