% !TEX root = ../bigmpi.tex

\section{Conclusions and Future Work}

In a time where 64 bit systems are widespread,
but C integer types remain 32 bits, describing large memory or file requests
will more frequently require the special handling BigMPI provides.
The exercise has also revealed several difficulties in the standard.
We have described fundamental issues with nonblocking
collective operations (reductions and both vector and neighborhood collectives)
that cannot be overcome using MPI-3 features.
The MPI Forum issued a challenge to consumers of MPI:
``prove to us that derived datatypes are insufficient''.
We believe this burden has been met and suggest the following features
be added to MPI in order to make holistic large-count support a reality.

We intend to drive the aforementioned tickets within the MPI Forum
in order to make complete large-count possible and efficient.
These features will be prototyped within MPICH and exploited by
BigMPI to provide that they are both necessary and sufficient.

A second area where ongoing development work is required is 
large-count tests that can be used to validate the count-safety of
MPI-3 implementations.  
