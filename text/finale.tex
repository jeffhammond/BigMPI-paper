% !TEX root = ../bigmpi.tex

\section{Conclusions and Future Work}

In a time where 64-bit systems are widespread
but C integer types remain 32 bits, describing large memory or file requests
will more frequently require the special handling that BigMPI provides.
The exercise has also revealed several difficulties in the MPI standard.
We have described fundamental issues with nonblocking
collective operations (reductions and both vector and neighborhood collectives)
that cannot be overcome by using MPI-3 features.
The MPI Forum issued a challenge to consumers of MPI:
``Prove to us that derived datatypes are insufficient.''
We believe this challenge has been met, and we suggest 
several features
that should be added to MPI in order to make holistic large-count support a reality.

Specifically, we intend to drive the aforementioned tickets
(see \S\ref{sec:mpi4}) within the MPI Forum
in order to make complete large-count possible and efficient.
These features will be prototyped within MPICH and exploited by
BigMPI to prove that they are both necessary and sufficient.
A second area where ongoing development work is required is 
large-count tests that can be used to validate the count-safety of
MPI-3 implementations.
Moreover, we plan to write a set of large-count tests for 
OpenSHMEM and GASNet.  The large-count tests of OpenSHMEM
will also serve as large-count tests for MPI-3, by virtue of 
OSHMPI~\cite{hammond2014implementing}.
